\documentclass[11pt]{article}
\usepackage[utf8]{inputenc}
\usepackage[a4paper,margin={2cm}]{geometry}

\usepackage{amsmath}
\usepackage{amssymb}
\usepackage{bm}
\usepackage{hyperref}

\usepackage{algorithm}
\usepackage[
    indLines=false,
    rightComments=false,
    beginComment=//~,
    beginLComment=//~,
]{algpseudocodex}

\newcommand{\R}{\mathbb{R}}
\newcommand{\Dt}{\Delta{}t}
\renewcommand{\Vec}{\bm}
\newcommand{\Norm}[1]{\left|#1\right|}

\newcommand{\AddEq}{\mathrel{+\!\!=}}
\newcommand{\SubEq}{\mathrel{-\!\!=}}

\newcommand{\MaxSpeed}{\textsf{MaxSpeed}}
\newcommand{\MaxAccel}{\textsf{MaxAccel}}

\begin{document}

Given position, velocity and acceleration, $\Vec{p}, \Vec{v}, \Vec{a} \in \R^{2}$, where
$\Vec{a} \neq \bm{0}$ Dishonored's movement system works as follows:
\begin{algorithm}
    \caption{\textsc{Movement}$(\Dt, s, f)$}
    \begin{algorithmic}[1]
        \State $\MaxSpeed = 400 \cdot s$
        \State $\MaxAccel = 2000 \cdot s$

        \State $\Vec{d} = \Vec{a}/\Norm{\Vec{a}}$

        \If{$\Norm{\Vec{a}} > \MaxAccel$}
            \State $\Vec{a} = \MaxAccel \cdot \Vec{d}$ \Comment{Limit acceleration to \MaxAccel}
        \EndIf

        \State $\Vec{v} \SubEq \Dt \cdot f \cdot (\Vec{v} - \Norm{\Vec{v}} \cdot \Vec{d})$
        \label{friction}\Comment{Friction}
        \State $\Vec{v} \AddEq \Dt \cdot \Vec{a}$ \label{acceleration}\Comment{Acceleration}

        \If{$\Norm{\Vec{v}} > \MaxSpeed$}
            \State $\Vec{v} = \MaxSpeed \cdot \Vec{v} / \Norm{\Vec{v}}$
            \Comment{Limit velocity to \MaxSpeed}
        \EndIf

        \State $\Vec{p} \AddEq \Dt \cdot \Vec{v}$ \Comment{Movement}
    \end{algorithmic}
\end{algorithm}

Here $\Dt, s, f \in \R$ represent respectively the delta time (in seconds), the speed scale, and
friction constant. The friction constant $f = 8$, and the speed scale $s$ depends on
the type of movement. Some relevant values are $s = 1$ when walking, $s = 1.5$ when sprinting, and
$s = 1.95$ when agility sprinting. We can generally assume $\Norm{\Vec{a}} = 2000 \leq \MaxAccel$.

Note that when $\Dt = \frac{1}{f} = \frac{1}{8}$, they cancel each other out during the friction
step (Line \ref{friction}), where we simply get
$\Vec{v} \SubEq \Vec{v} - \Norm{\Vec{v}} \cdot \Vec{d}$, which is equivalent to
$\Vec{v} = \Norm{\Vec{v}} \cdot \Vec{d}$, i.e., all previous speed gets redirected in the direction
we are currently moving. This property is unique to 8 FPS.

It is possible to achieve something similar at other frame rates if we take into account the acceleration
step (Line \ref{acceleration}), by solving the equation
\begin{equation*}
    m \cdot \Vec{u} = \Vec{v} -  \Dt \cdot f \cdot (\Vec{v} - \Norm{\Vec{v}} \cdot \Vec{d})
                              + \Dt \cdot 2000 \cdot \Vec{d},
\end{equation*}
for $m$ and $\Vec{d}$ where $\Norm{\Vec{d}} = 1$ in order to move in direction $\Vec{u}$.

If we make the simplifying assumption that $\Vec{v} = \MaxSpeed \cdot \Vec{w}$ for some unit vector
$\Vec{w}$, we can further simplify the equation to

\begin{equation*}
    \begin{split}
        m \cdot \Vec{u} &= \MaxSpeed \cdot \Vec{w}
                                    -  \Dt \cdot f \cdot \MaxSpeed \cdot (\Vec{w} - \Vec{d})
                                    + \Dt \cdot 2000 \cdot \Vec{d} \\
                        &= 400 \cdot s \cdot \Vec{w}
                                    - \Dt \cdot f \cdot 400 \cdot s \cdot (\Vec{w} - \Vec{d})
                                    + \Dt \cdot 2000 \cdot \Vec{d} \\
    \iff
        \frac{m}{400 \cdot s} \cdot \Vec{u} &= \Vec{w} - \Dt \cdot f \cdot (\Vec{w} - \Vec{d})
                                                       + \frac{5}{s} \cdot \Dt \cdot \Vec{d}. \\
    \end{split}
\end{equation*}
We let $m' = \frac{m}{400 \cdot s}$ and $a = \frac{5}{s}$, and end up with the final equation
\begin{equation*}
    m' \cdot \Vec{u} = \Vec{w} - \Dt \cdot f \cdot (\Vec{w} - \Vec{d}) + a \cdot \Dt \cdot \Vec{d}.
\end{equation*}
It is easy to rearrange in order to find
\begin{equation*}
    \Vec{d} = \frac{m' \cdot \Vec{u} + (\Dt \cdot f - 1) \cdot \Vec{w}}{\Dt \cdot (a + f)}.
\end{equation*}
Solving for $m'$ is harder but yields
\begin{equation*}
    m' = \sqrt{(\Dt \cdot (a + f))^{2} - ((\Dt \cdot f - 1) \cdot \Vec{u} \cdot \Vec{w'})^{2}}
       - (\Dt \cdot f - 1) \cdot \Vec{u} \cdot \Vec{w},
\end{equation*}
where $\Vec{w'} = (-w_{2}, w_{1})$.

If we want to simply turn an absolute amount, we can fix $\Vec{w} = (1, 0)$, which simplifies
the above equations to
\begin{equation*}
    \Vec{d} = \frac{m' \cdot \Vec{u} + (\Dt \cdot f - 1, 0)}{\Dt \cdot (a + f)},
\end{equation*}
and
\begin{equation*}
    m' = \sqrt{(\Dt \cdot (a + f))^{2} - ((\Dt \cdot f - 1) \cdot u_{2})^{2}}
       - (\Dt \cdot f - 1) \cdot u_{1}.
\end{equation*}

\end{document}
